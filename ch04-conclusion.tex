\chapter{Conclusion and Outlook}

In this thesis I presented an integration of the Object Tracker for the HoloLens 2 ecosystem. I presented the Sensor API, an interface to get data streams from various HoloLens 2 sensors via sensor interfaces which use standard Windows APIs and Research Mode. Because I don't use any other libraries than standard Windows APIs and Research Mode in the Sensor API, it can easily be used for different projects with the similar requirements.

I then used these raw data streams to incorporate the Object Tracker into a mixed reality application which provides a simple user interface to interact with the Object Tracker. I also used the Cannon library to display bricks tracked by the Object Tracker in the holographic space which allows for the user to accurately build a simple brick tower by hand.

The current application can be improved further in a couple of different ways. For one, the Object Tracker could make use of the depth camera of the HoloLens 2. The Sensor API already provides an interface to retrieve this data however it is not yet used by the Object Tracker.

Another point of improvement is Research Mode itself. While it grants access to low-level sensor streams which are otherwise not available, the API is not yet optimized for applications working in real-time. For one, IMU measurements retrieved via Research Mode always lag a couple of tenths of a second behind real-time. This introduces lag into the application which makes it more difficult to perform real-time taks. While it is possible to retrieve a lot of measurement samples per second for every IMU sensor, these measurement samples arrive in batches at low frame rates which reduces some of the advantages of having many samples per second available. If Microsoft decides to improve Research Mode in the future, such changes would have a large impact on real-time object tracking.

Finally, the performance of the Object Tracker on the HoloLens 2 is not as good as on other platforms because of limited computational power. Some of the optimization steps take a lot of time which can result in delayed adjustments and general instability. This could be improved by looking at ways to optimize the Object Tracker further, especially in regard of the ecosytem of the HoloLens 2. It is also to be expected that future generations of mixed reality headsets will have more computational power which will of course diminish this effect.